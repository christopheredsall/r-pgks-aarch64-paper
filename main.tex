\documentclass{article}
\usepackage[utf8]{inputenc}
\usepackage{listings}

\title{A Survey of R Package Installabilty on 64-bit ARM Architecture aarch64 Using Continuous Integration}
\author{Edsall, Christopher\\
  \texttt{chris.edsall@bristol.ac.uk}
  \and
  Rutterford, Ainsley\\
  \texttt{ar16478@bristol.ac.uk}
}

\date{August 2019}

\begin{document}

\maketitle

\section{Introduction}
The R programming language is the preeminent open source language in the domain of statistics. A large part of the popularity of the language is due to the extensive library of packages hosted on the comprehensive R archive network (\textsc{CRAN}). The \textsc{CRAN} website lists the package build status on a number of platforms but these do not include \texttt{aarch64}. A number (n=XXX) of users at our institution use R, what proprotion of them could make use of 64-bit ARM platforms?

\section{Testing Package Installability}

\begin{lstlisting}[language=R, caption=Testing that package \texttt{tidyr} can be installed and loaded]
if (!require('tidyr')) {
install.packages('tidyr',
                 repos='https://www.stats.bris.ac.uk/R/')
}
library('tidyr')
\end{lstlisting}


\section{Generation of the List of Packages}

\section{Continuous Integration Platforms Supporting aarch64}

The criteria for choosing a continuous integration platform is that the "runner" or "agent" had to support \texttt{aarch64}.

\begin{table}[]
\begin{tabular}{lll}
\cline{1-3}
Service & Usable on aarch64 & Comments\\ \cline{1-3}
Travis & No & Runs via emulation on their infrastructure\\
GitLab & No & aarch64 runner not yet production\\
Azure Pipelines & No & ---\\
Buildkite & \textbf{Yes} & --- \\
Buildbot & Probably & Not evaluated due to time \\ \cline{1-3}
\end{tabular}
\caption{List of continuous integration services, status as at Sept 2019}
\label{tab:ci-services}
\end{table}

\section{Results}



\end{document}
