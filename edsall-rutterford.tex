% !TeX root = RJwrapper.tex
\title{A Survey of R Package Installabilty on 64-bit ARM Architecture aarch64 Using Continuous Integration}
\author{by Christopher Edsall and Ainsley Rutterford}

\maketitle

\abstract{
A large part of the popularity of the R language is due to the extensive
library of packages hosted on repositories such as the comprehensive R
archive network (\textsc{CRAN}) and BioConductor. The \textsc{CRAN} website
lists the package build status on a number of platforms but these do not
include \texttt{aarch64}. A number (n=XXX) of users at our institution use R
on the high performance computing facilities, what proportion of them could
make use of the 64-bit ARM platforms? This paper proposes a method to test
and shows results for an important subset of the total population of
libraries 95\% of them can be installed "out of the box".
}

\section{Introduction}

The R programming language\citep{R:Ihaka+Gentleman:1996} is the preeminent
open source language in the domain of
statistics\citep{muenchen2012popularity}. The interpreter has been ported to
many computer architectures including the \textsc{ARM} aarch64. To extend
the functionality of the language packages can be installed.

R packacge compiling

Other aarch64 HPC environment.

test vs X86-64

\section{Testing Package Installability}

\code{
if (!require('tidyr')) {
install.packages('tidyr',
                 repos='https://www.stats.bris.ac.uk/R/')
}
library('tidyr')
}

Failures: bioconductor not installed, missing operating system packages, missmatch with interpreter versions. PAckage removed from CRAN (RNeo4j)

\section{Generation of the List of Packages}

The total number of packages in the \textsc{CRAN} as of 27 August 2019 was
14857 and the number in BioCondcutor was 1741. As a method of reducing the
computation required we decided to select a subset used by our users. We
performed a search of the filesystem on our two largest Intel based clusters
to find files named \texttt{*.r} and \texttt{*.R}.

We excluded individual's own pacakges and those that are not on
\textsc{CRAN} or BioConductor.

\section{Continuous Integration Platforms Supporting aarch64}

Given that the list of packages to be tested is very large and that it would need to be retested following system changes made to resolve build failures, we elected to use Continuous Integration (\textsc{CI}) to run the tests.

The criteria for choosing a \textsc{CI} platform is that the "runner" or "agent" had to support \texttt{aarch64}.

\begin{table}[htbp]
\begin{tabular}{lll}
\cline{1-3}
Service & Usable on aarch64 & Comments\\ \cline{1-3}
Travis & No & Runs via emulation on their infrastructure\\
GitLab & No & aarch64 runner not yet production\\
Azure Pipelines & No & ---\\
Buildkite & \textbf{Yes} & --- \\
Buildbot & Probably & Not evaluated due to time \\ \cline{1-3}
\end{tabular}
\caption{List of continuous integration services, status as at Sept 2019}
\label{tab:ci-services}
\end{table}

\section{Systems Tested}

\begin{table}[h]
\begin{tabular}{lllll}
\cline{1-5}
System Name & Vendor & Model & OS & R Module \\ \cline{1-5}
Isambard & Cray & XC50 & SUSE 15 & cray-R/3.4.2 \\
Catalyst & HPE & Apollo 70 & SUSE 12 & lang/r/3.6.0-gcc \\ \cline{1-5}
\end{tabular}
\caption{HPC clusters tested }
\label{tab:systems}
\end{table}

\section{Results}

As an initial test we took the list of packages (n=44) used by one user. The results were

\begin{table}[h]
\begin{tabular}{lll}
\hline
System & PASS & FAIL \\ \hline
Isambard & 19 (43\%) & 25 (57\%) \\
Catalyst & 42 (95\%) & 2 (5\%) \\ \hline
\end{tabular}
\caption{Initial results for 44 pacakges}
\label{tab:results-one-user}
\end{table}

\section{Further Work}

\begin{itemize}
    \item Test on AWS EC2 A1 Instances using the AWS Graviton processors
    \item Roll in to CRAN
\end{itemize}

\section{Acknowledgments}

The authors wish to thank the University of Bristol Advanced Computing
Research Centre (\textsc{ACRC}) for access to the aarch64 clusters. This
work used the Isambard UK National Tier-2 HPC Service
(http://gw4.ac.uk/isambard/) operated by GW4 and the UK Met Office, and
funded by EPSRC (EP/P020224/1). The Catalyst UK programme is supported by
ARM, HPE and SUSE.

\bibliography{edsall-rutterford}

\address{Christopher Edsall\\
  University of Bristol\\
  31 Great George Street\\
  Bristol\\
  BS1 5QD\\
  United Kingdom\\
  ORCiD: 0000-0001-6863-2184\\
  \email{chris.edsall@bristol.ac.uk}}

\address{Ainsley Rutterford\\
  University of Bristol\\
  31 Great George Street\\
  Bristol\\
  BS1 5QD\\
  United Kingdom\\
  (ORCiD if desired)\\
  \email{ar16478@bristol.ac.uk}}
